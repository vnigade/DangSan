\section{Related work}
\label{sec:related}

Besides the different memory shadowing~\cite{akritidis2008preventing,akritidis2009baggy,younan2015freesentry}
and tree-based approaches~\cite{haller2013mempick,lee2015preventing} already discussed within this paper (in Sections~\ref{sec:introduction}
 and~\ref{sec:applications}) there is one system aiming
to solve the metadata tracking problem in a similar manner using a variable compression ratio. The
custom memory allocator implemented within the Sanitizer library of LLVM offers a metadata retrieval system,
which shares characteristics with \projectname{}. The system has been used within CaVer~\cite{lee2015type}
to track type information for heap objects, but is not described in other contexts.

This memory allocator reserves a specific part of the address-space
,as a Region, for each size category defined within the system. Regions are then split into slots of equal
size, with each slot containing a single object. This allows the allocator to associate each slot/object
with a singular piece of metadata, accessed by identifying the slot index corresponding to any pointer.
The Region is also easily identified from the pointer, since the organization of the address-space into
Regions happens at compile-time. This systems shares similarities with \projectname{}, with the primary difference being
the granularity of the memory blobs associated to a certain allocation size. \projectname{} only requires
memory pages to be uniform, which is already assured by the general purpose allocator \emph{tcmalloc}. This custom allocator
on the other hand restricts allocations within a certain Region, whose addresses are predefined at compile-time.
The result is a significant reduction of the potential for address-space randomization, with the address bits
of each Region being predictable by the attackers. Furthermore, the current implementation
also enforces a highly predictable allocation pattern within the Region itself with no ability to release
memory back to the system. The effort to add these highly desirable features to the current design is
unclear at this point. These concerns are of course irrelevant when using the allocator within the desired
context of software testing, but they are key characteristics for a general purpose production allocator.

While the design works efficiently for heap objects, Lee et al.~\cite{lee2015type} argue
that it is not directly applicable to stack and global memory. Their solution to the problem involves
integrating a tree-based metadata tracking for these memory types.
As far as we are aware, \projectname{} is the first design to combine uniformity across memory types with all the advantages
brought forward by variable compression ratio metadata tracking.
