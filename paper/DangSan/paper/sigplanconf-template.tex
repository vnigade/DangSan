%-----------------------------------------------------------------------------
%
%               Template for sigplanconf LaTeX Class
%
% Name:         sigplanconf-template.tex
%
% Purpose:      A template for sigplanconf.cls, which is a LaTeX 2e class
%               file for SIGPLAN conference proceedings.
%
% Guide:        Refer to "Author's Guide to the ACM SIGPLAN Class,"
%               sigplanconf-guide.pdf
%
% Author:       Paul C. Anagnostopoulos
%               Windfall Software
%               978 371-2316
%               paul@windfall.com
%
% Created:      15 February 2005
%
%-----------------------------------------------------------------------------


\documentclass[preprint]{sigplanconf}

% The following \documentclass options may be useful:

% preprint      Remove this option only once the paper is in final form.
% 10pt          To set in 10-point type instead of 9-point.
% 11pt          To set in 11-point type instead of 9-point.
% numbers       To obtain numeric citation style instead of author/year.

\usepackage{amsmath}

\newcommand{\cL}{{\cal L}}

\begin{document}

\special{papersize=8.5in,11in}
\setlength{\pdfpageheight}{\paperheight}
\setlength{\pdfpagewidth}{\paperwidth}

\conferenceinfo{CONF 'yy}{Month d--d, 20yy, City, ST, Country}
\copyrightyear{20yy}
\copyrightdata{978-1-nnnn-nnnn-n/yy/mm}
\copyrightdoi{nnnnnnn.nnnnnnn}

\newcommand{\projectname}[0]{DangSan}
% Uncomment the publication rights you want to use.
%\publicationrights{transferred}
%\publicationrights{licensed}     % this is the default
%\publicationrights{author-pays}

%\titlebanner{banner above paper title}        % These are ignored unless
%\preprintfooter{short description of paper}   % 'preprint' option specified.

%\title{DangSan: A mitigation technique to prevent dangling pointers exploit}
%\subtitle{Subtitle Text, if any}
\title{DangSan: Practical Dangling Pointers Prevention}
\authorinfo{Vinod Vijay Nigade}
           {Vrije Universiteit Amsterdam}
           {vinod.nigade@gmail.com}
%\authorinfo{}
%           {}
%           {}

\maketitle

\begin{abstract}
% What is the problem?
Many system applications are still developed using unsafe languages like C/C++. Memory corruption due to dangling pointers in these applications can lead to Use-after-Free (Double-Free) vulnerabilities. Many Use-after-Free detection schemes and systems have been proposed in the previous work. These systems either incur high performance overhead or have limited applicability.
% What we did?
This paper proposes \projectname{}, a simple and efficient lock-less system to prevent Use-after-Free exploits during runtime. \projectname{} uses LLVM compiler instrumentation pass to insert run-time pointer tracking functions. During run-time, most of the pointers that point to the object are tracked. These pointer values are set to invalid addresses when the object is freed. \projectname{} is based on efficient metadata management framework, \metalloc{}~\cite{istvan2016metalloc}.
% How we did?
% what we achieved
 It has moderate performance overhead. For SPEC2006 benchmarks, it has $43.9\%$ average (geometric mean) run-time overhead when only heap pointers are tracked. For widely used web-servers (\texttt{nginx}, \texttt{httpd} and \texttt{lighttpd}), it has moderate throughput degradation of $12.8\%$ and negligible service latency overhead. Moreover, \projectname{} successfully prevented recently discovered Use-after-Free vulnerabilities in widely used complex software.
\end{abstract}


%\category{CR-number}{subcategory}{third-level}
% general terms are not compulsory anymore,
% you may leave them out
%\terms
%term1, term2
%\keywords
% Check this Koustubha???
%use-after-free, SPEC2006


\section{Background}
1) Instrumentation phase
2) Implementation of run-time using FreeSentry approach
3) Comparison of with lock and without lock
4) Need to go for other approaches

\section{System Design}

\section{Implementation}

\section{Evaluation}
\subsection{Performance Analysis}
\subsection{Correctness}

\section{Related Work}

\section{Conclusion}

\appendix
\section{Appendix Title}

This is the text of the appendix, if you need one.

\acks

Acknowledgments, if needed.

% We recommend abbrvnat bibliography style.

\bibliographystyle{abbrvnat}

% The bibliography should be embedded for final submission.

\begin{thebibliography}{}
\softraggedright

\bibitem[Smith et~al.(2009)Smith, Jones]{smith02}
P. Q. Smith, and X. Y. Jones. ...reference text...

\end{thebibliography}


\end{document}
