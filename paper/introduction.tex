\section{Introduction}
\label{sec:introduction}

Many common software security hardening solutions need to maintain and look up
memory metadata at runtime. Examples include bounds information to validate
array references~\cite{akritidis2008preventing,akritidis2009baggy}, type information to validate cast operations~\cite{lee2015type}, solutions that prevent use-after-free exploits~\cite{lee2015preventing,younan2015freesentry},
and object pointer information to perform garbage collection~\cite{rafkind2009precise}.
While newer programming languages can often track
such metadata in-band using fat pointers, previous efforts to implement
in-band metadata management in systems programming languages, such as C or C++,
have found limited applicability due to poor ABI compatibility and nontrivial overhead~\cite{dhurjati2006backwards}.

The alternative solution is to associate metadata information with the memory objects themselves, assuming we have a mechanism to map pointers to the appropriate metadata.
Such a  primitive is the key to implementing  modern metadata management schemes,
%% given 
but it is also challenging because it needs to support all the possible memory objects (heap objects allocated
with \texttt{malloc()}, globals, and stack objects)
as well as minimize the performance and memory impact
of metadata update and lookup operations.

Minimizing the impact of update operations
is challenging,
because allocation and deallocation of memory objects and their metadata occurs frequently during execution. 
%% many memory objects (e.g., stack objects) and their metadata are
%% frequently allocated/deallocated during the execution.
Minimizing the impact of lookups is also
challenging, since metadata lookup must be able to support interior pointers
into nested classes, structures, and arrays---dictating support for range queries and disqualifying the use of space- and time-efficient hash tables.
\looseness=-1

Current state-of-the-art software hardening projects all rely on tailored, mostly one-off 
solutions for metadata
management, but none of them simultaneously achieves low lookup and update impact in all cases. As a result, none
of them provides a generic solution. 
Common approaches include tree-based metadata handling and memory shadowing.
 
Tree-based approaches~\cite{haller2013mempick,lee2015preventing} store
an interval node for each allocated object according to to its bounds.
Unfortunately, tree lookups can result in a prohibitive performance hit,
as the tree depth is frequently in the double digit range (more than 1,024 memory objects).
The lookup time is also unpredictable, as it varies with the object count.
As a result,  tree-based systems are unsuitable
for most production situations.
\looseness=-1

Traditional memory shadowing, in turn, relies on a fixed
pointer-to-metadata mapping~\cite{akritidis2008preventing,akritidis2009baggy,younan2015freesentry},
The key design choice for this approach is the \emph{metadata compression ratio}.
The metadata compression ratio represents the number of metadata bytes that need to
be tracked for each data byte.
For example, assume we store one byte of metadata for each block of eight bytes.
In this case the compression ratio is $\frac{1}{8}$. If we have a pointer $p$ and
an array of metadata starting at address $q$, we can compute a pointer
to the metadata for object as $\frac{1}{8} p + q$.
This way, metadata can be located very efficiently.
However, choosing the appropriate
compression ratio is difficult, as it enforces a minimum alignment on every memory allocation.
Small compression ratios result in inflated metadata size and a large tracking overhead,
while large compression ratios result in significant memory fragmentation.
In practice, memory management systems typically only guarantee alignment
up to 8 or 16 bytes. This means that to keep the compression ratio reasonably
small only a single byte of metadata is supported~\cite{akritidis2008preventing,akritidis2009baggy}.
Even then, this approach introduces prohibitive initialization time and
memory overhead for large objects in case multi-byte metadata is needed.
Finally, recent approaches rely on custom
allocators to reduce the impact of memory shadowing on the heap, but
cannot support efficient and comprehensive metadata management
including more performance-sensitive objects on the stack~\cite{lee2015type}.

In this paper, we propose \projectname{}, a new metadata memory management scheme
based on an efficient and comprehensive variable memory shadowing strategy. Our
strategy builds on recent developments in heap~\cite{ghemawat2009tcmalloc} and stack~\cite{kuznetsov2014cpi} organizations
to implement a variable and uniform pointer-to-shadow mapping
and significantly reduce the performance and memory impact of metadata management.
Our results show that \projectname{} is practical and can support efficient
whole-memory metadata management for several software security hardening solutions.

Summarizing, we make the following contributions:

\begin{itemize}
%\setlength\itemsep{1.5em}
\item We propose a new memory metadata management scheme that supports interior pointers
      and is time- and space- efficient in both lookups and updates across all memory object types.

\item We present a prototype implementation termed \projectname{},
      which demonstrates that efficient and comprehensive metadata management is feasible and widely applicable in practice.

\item We present an empirical evaluation showing that \projectname{} incurs a run-time performance overhead of just 1.2\% on SPECint2006.
  
\end{itemize}
